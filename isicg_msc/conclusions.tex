\chapter{Conslusions}\label{conclusions}

In this work we explored the problem of information retrieval in the field of image processing. We studied two widely used image descriptors applied in several applications while we chose to evaluate them within several setups in the scope of our building retrieval framework.
We also proposed a novel building database featuring a number of buildings with architectural variations, captured in different illumination conditions under individual viewpoints. 
The goal of this work was to construct a challenging subset of experiments in order to extensively evaluate the aforementioned methods. Feature extraction and image matching methodologies were used to forge a ground truth of experimental results in order to extract knowledge through specific measures.
More specific, we carried out the experiment methodology in two individual scenarios, in detection and retrieval. Moreover we conducted these scenarios in a twofold process, defining two subset of experiments: a) selected a handpicked subset of 90 photos, which are actually all captures from 6 different buildings and b) selected a handpicked subset consisted of 60 photos, which are the frontal views of all buildings. Moreover, concerning the descriptors, each of the aforementioned subset delivered in two sets of input parameters.
In case of detection, for first subset of experiments, SURF features in default setting, observed in higher performance. Also peak value of the above measures recorded in one rank higher inlier threshold. On the other hand, in case of retrieval, SIFT features in default setting, proved slightly more appropriate.
Overall SURF features, delivered the experiments faster in throughput time.
Aside from Vyronas database which proved a challenging dataset, the proposed system, recorded in remarkable results when we contaminated the dataset with a number of 1000 and 5000 of Oxford buildings. The reduction of performance was acceptable thus we didn't notice any extravagant case in both scenarios.
Experimental results led us to interesting visual data extraction which are used for demonstration through the proposed web platform. RetBul platform is enhanced with state-of-the-art open source technologies, optimized to achieve the finest performance.


\section{Future Work}

In this thesis, feature extraction techniques have proven liable, though, an exploration and use of other blob based descriptors can be proved vital. Moreover, testing more parameters of each descriptor extraction method is a significant step towards the improvement of the overall system’s performance. On the other hand, the current work can be further extended by involving a bag-of-visual-words framework. The problem then is to properly quantize the descriptors in order to construct an efficient visual vocabulary. Moreover, a weighting scheme, e.g. like tf-idf, could also be incorporated in order to take into account the appearance frequencies of the visual words. It is clear that describing an image by a set of visual words would significantly decrease both the memory requirements as well as the computational effort of the retrieval process, especially if applied in the proposed web application.