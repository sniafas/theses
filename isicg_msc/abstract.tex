
\thispagestyle{empty}

\begin{center}
\Large
{\bf \textgreek{ΕΥΧΑΡΙΣΤΙΕΣ}}\\[15mm]
\end{center}

\textgreek{ Θα ήθελα να εκφράσω τις ευχαριστίες μου καταρχήν προς τον επιβλέποντα της διπλωματικής μου εργασίας κ. Κεσίδη Αναστάσιο για τα εποικοδομητικά του σχόλια και τον ορθό τρόπο συγγραφής που μου επέδειξε καθώς και για την ειλικρινέστατη συνεργασία που είχαμε.
Επίσης, οφείλω πολλά στο φίλο και μέντορά μου Δρ. Ευάγγελο Σπύρου, που βρισκόταν πάντα κοντά μου τα τελευταία αυτά χρόνια, που με την καθοδήγηση και την αμέριστη βοήθειά του, έφτασα στο σημείο που βρίσκομαι σήμερα.
Ευχαριστώ την οικογένειά μου που με τη στήριξη και την υπομονή τους με βοηθήσαν να γίνω ένας ολοκληρωμένος άνθρωπος. Τέλος τους φίλους μου, που καταφέραμε να κρατηθούμε όλα αυτά τα χρόνια και στέκονται δίπλα μου στις δύσκολες στιγμές όταν τους έχω ανάγκη.\\ \\Σταύρος Νιάφας\\
Σεπτέμβριος 2016
}

\newpage

\begin{center}
\Large
{\bf Abstract}\\[15mm]
\end{center}

During the last few years, the production of digital content has been continuously increasing. Digital cameras have been integrated to computers, mobile phones and tablets and have become interdependent to many daily activities. As a result, the research fields of  digital image processing and computer vision have benefited from content availability and many new research topics have arisen. The goal of this Thesis is to tackle the problem of building recognition, i.e., given a query image of a specific building, to retrieve images of the same building within a database. To this goal, we choose to follow a traditional approach of content-based image retrieval. We first extract visual features and then, by imposing geometrical constraints on them we estimate a measure of similarity between two given images. We face both aspects of the aforementioned problem, i.e., detection and retrieval. Moreover, we construct a novel building database, consisting of a set of views from a large number of heterogeneous buildings, under several lighting conditions. We use this dataset to evaluate several setups of the proposed approach. Finally, we create a fully functional web-based image retrieval platform, using state-of-the-art technologies, whose purpose to facilitate experiments while also to serve for demonstration and educational purposes.  


\vspace{50mm}
