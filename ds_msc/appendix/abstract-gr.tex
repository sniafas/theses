\selectlanguage{greek}
\chapter*{Περίληψη}
\lettrine[lines=2]{\textbf{Σ}}{} κοπός αυτής της εργασίας είναι η αναγνώριση και ποσοτικοποίηση της αισθητικής σε εικόνες για μια συγκεκριμένη φωτογραφική τεχνοτροπία (Μποκέ), η οποία χαρακτηρίζεται από το επίπεδο στο βάθος πεδίου. Το πρόβλημα αντιμετωπίστηκε από τη σκοπιά της εκτίμησης ποιότητας της αισθητικής στην εικόνα, χρησιμοποιώντας βαθιά ενεργή μάθηση.

Αρχικά παρουσιάζονται στον αναγνώστη οι βασικές έννοιες της φωτογραφίας. Στη συνέχεια θέτουμε την αισθητική ως φιλοσοφικό ερώτημα και ορίζουμε το πρόβλημα της ποσοτικοποίησης με υπολογιστικές μεθόδους, βασιζόμενοι σε θεμελιώδεις θεωρίες της ανθρώπινης αντίληψης και ψυχολογίας.

Γενικά, μελετάμε τον τομέα της τεχνητής νοημοσύνης και της υπολογιστικής όρασης. Πιο συγκεκριμένα, εφαρμόζουμε μεθόδους μηχανικής και βαθιάς μηχανικής μάθησης και τις συνδυάζουμε με τεχνικές ενεργής μάθησης. 
Επιπρόσθετα, χρησιμοποιήσαμε ένα σύνολο δεδομένων υπερυψηλής ανάλυσης και παρουσιάζουμε ένα νέο καινοτόμο σύνολο, που δημιουργήσαμε με επισημάνσεις υψηλής ποιότητας βασιζόμενοι σε γνώσεις φωτογραφίας και μεθόδους ενεργής μάθησης.

Αρχικά, παρουσιάζουμε μια πειραματική διαδικασία που αποσκοπεί στη δημιουργία ενός ταξινομητή ικανού να αναγνωρίσει το βάθος πεδίου και στη συνέχεια εφαρμόζουμε τεχνικές ενεργής μάθησης, οι οποίες συγκριτικά με τις τυπικές μεθόδους, στοχεύουν στην ελαχιστοποίηση του χρόνου επισήμανσης καινούριων δειγμάτων, ενώ παράλληλα συμβάλλουν στην ταχύτερη αύξηση της απόδοσης του ταξινομητή.

Τέλος, σημειώνεται πως στην παρούσα διατριβή έχει δοθεί ιδιαίτερη έμφαση στη μεθοδολογία και τη δημιουργία του συνόλου δεδομένων, το οποίο χρησιμοποιήθηκε για την εκπαίδευση, την εκτίμηση των αλγορίθμων και τη γενική προσέγγιση του προβλήματος. 
Για το σκοπό αυτό δημιουργήθηκαν δύο διαφορετικά επισημασμένα σύνολα, το ένα με αποκλειστική παρέμβαση του ανθρώπινου παράγοντα και το δεύτερο με συνδυασμό μεθόδων ενεργής μάθησης.\\
\textbf{Θεματική περιοχή}: Αξιολόγηση αισθητικής εικόνας\\
\textbf{Λέξεις κλειδιά}: Ταξινόμηση εικόνων, αισθητική εικόνων, ενεργή μάθηση
