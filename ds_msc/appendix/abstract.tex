\selectlanguage{english}
\chapter*{Abstract}
\fancypar{T}his thesis aims to recognize and quantify image aesthetics in a certain photography style (Bokeh) which is characterized from the level in depth of field. The problem is treated as a task related to image aesthetics quality assessment approached with deep active learning.

Starting with, the reader is introduced to photography basics. We refer to aesthetics as a general philosophical question and state the problem of quantifying aesthetics with computational methods based on fundamental principles of human perception and psychology.

In general, we study the domains of artificial intelligence and computer vision. More specifically we apply machine and deep learning techniques and combine them with active learning methods. Additionally, we utilised a super high resolution data set and present a novel one, created with high quality labels based on photography domain knowledge and active learning strategies.

Firstly, we present an experimental process to generate a classifier able recognize the depth of field in images and secondly apply active learning practices which, compared to regular training methods, aim to effectively reduce annotation costs and at the same time contribute to faster increase in classifier's performance.

Finally, in this thesis, extra emphasis has been given to develop a methodology in order to create a novel data set. We utilised it in order to train and evaluate the machine learning algorithms and generally to approach the problem.
To this end, two different annotated data sets were created. One with exclusive human input and a second combined with active learning methods.\\
\textbf{Subject Area}: Image aesthetics quality assessment\\
\textbf{Keywords}: Image classification, image aesthetics, active learning
