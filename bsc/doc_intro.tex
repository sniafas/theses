
\section{Εισαγωγή}

\subsection{H τεχνολογία WCE}
 
Στην σύγχρονη ιατρική, μοντέρνες τεχνικές ενδοσκόπησης έχουν φέρει επανάσταση στη διάγνωση και τη θεραπεία ασθενειών
του ανώτερου και κατώτερου γαστρεντερικού (οισοφάγος, στομάχι, δωδεκαδάχτυλο, λεπτό και παχύ έντερο). 
Μια από τις νεότερες και πιο επαναστατικές τεχνολογίες, που βοηθούν και επεκτείνουν τις υπάρχουσες τεχνικές διάγνωσης στον τομέα της ενδοσκόπησης του γαστρεντερικού σωλήνα, όπως η EGD~\cite{egd} και η κολονοσκόπηση,
είναι η WCE (wireless capsule endoscopy). H τεχνολογία αυτή είναι χρήσιμη, όταν η ασθένεια αφορά στο λεπτό έντερο, όπως η νόσος του Crohn (Crohn's disease)~\cite{crohns} ή στο ανώτερο γαστρεντερικό, όπως το πεπτικό έλκος (peptic ulcers)~\cite{peptic}. \par
Η τεχνολογία WCE, με τη χρήση μιας κάψουλας, σε μέγεθος μεγάλης βιταμινούχου κάψουλας, ενσωματώνει μια μικροσκοπική έγχρωμη βιντεοκάμερα, φακό, μπαταρία και πομπό. Χορηγείται στον ασθενή από του στόματος, με την κατάποση φθάνει στον
οισοφάγο και στη συνέχεια αποτυπώνει και καταγράφει το εσωτερικό του γαστρεντερικού σωλήνα. Στο Σχήμα~\ref{fig:capsule_endoscopy}, φαίνονται οι διαστάσεις της κάψουλας που χορηγείται στον ασθενή και το Σχήμα~\ref{fig:capsule_fig} αποτυπώνει
ένα στιγμιότυπο από το εσωτερικού του εντέρου.\par
Η παρακολούθηση της πορείας, που διαγράφει η ειδική κάψουλα μέσα στο ανθρώπινο σώμα, υλοποιείται ουσιαστικά από ένα πλήθος εξωτερικών αισθητήρων, 
ικανών να εκτιμήσουν, με κάποιες αποκλίσεις, την πορεία και τη θέση της κάμερας~\cite{spyrou2013panoramic,spyrou2012homography,capsend}. 

\newpage

\begin{figure}[!ht]
\begin{minipage}[b]{1.0\linewidth}
\centering
\includegraphics[scale=1.5]{../attachments/capsule.jpg}
\end{minipage}
\caption{H ενδοσκοπική κάψουλα.}
\label{fig:capsule_endoscopy}  %%% μονα τους
\end{figure} 

\begin{figure}[!ht]
\begin{minipage}[b]{1.0\linewidth}
\centering
\includegraphics[scale=0.4]{../camera_ready_figs/Fig_14_a.png}
\end{minipage}
\caption{Εικόνα από τον αυλό του εντέρου εκ των έσω με ενδοσκοπική κάψουλα.}
\label{fig:capsule_fig}
\end{figure} 



\subsection{Προβλήματα της WCE}
 
   Ωστόσο, παρά το ότι η κάψουλα παρέχει με τον καλύτερο δυνατό τρόπο εικόνες εκ των έσω από τον αυλό του πεπτικού σωλήνα, υπάρχουν περιορισμοί και προβλήματα κατά τη χρήση της~\cite{website:medicinenet}.
   
     \begin{itemize}
    
   \item Η γρήγορη μετάβαση της κάμερας από ορισμένα σημεία του εντέρου έχει αρνητικές επιπτώσεις στα εξαγόμενα δεδομένα, π.χ. θαμπή εικόνα, 
 αδυναμία αναγνώρισης των εικόνων, λόγω ασθενούς σήματος.
   \item Σε διαστήματα που η προώθηση της κάψουλας είναι αργή ή σταματά, η κάψουλα παρέχει δεδομένα μόνο από τη συγκεκριμένη περιοχή του εντέρου, ενώ επιπρόσθετα σε περιπτώσεις πολύωρης κατακράτησης υπάρχει κίνδυνος να 
τελειώσει η ενέργεια της μπαταρίας της, που έχει διάρκεια ζωής περίπου οκτώ ώρες.
   \item Εν κατακλείδι, η μελέτη των εικόνων που λαμβάνονται δυνατόν να είναι εξαιρετικά χρονοβόρα, καθώς συλλέγονται δεκάδες χιλιάδες φωτογραφίες κατά τη διάρκεια της εξέτασης με την ενδοσκοπική κάψουλα. 
 
 
  \end{itemize}

   
\subsection{Τρόπος καταγραφής και ελέγχου των αποτελεσμάτων}


H κάψουλα χορηγείται αναλόγως ενδείξεων στον ασθενή, που του ζητάτε να την καταπιεί, όπως προαναφέρθηκε. Με αυτόν τον τρόπο η κάψουλα διέρχεται από τον οισοφάγο, περνά στο στομάχι
και με τις περισταλτικές κινήσεις του πεπτικού σωλήνα προωθείται περιφερικότερα στο λεπτό έντερο μέχρι να καταλήξει στο παχύ έντερο και τελικά να αποβληθεί από τον ανθρώπινο σώμα στα επόμενα 24ωρα, 
τυπικά σε διάστημα 5-8 ωρών. Καθ' όλη τη διάρκεια της διαδρομής, η κάψουλα καταγράφει χιλιάδες στιγμιότυπα και τα εκπέμπει ασύρματα σε μία συσκευή εγγραφής.
Ο εντοπισμός και η παρακολούθηση της κάμερας στο εσωτερικό του αυλού του πεπτικού σωλήνα πραγματοποιείται από αισθητήρες ραδιοσυχνοτήτων (Radio Frequency), 
που είναι προσαρμοσμένοι στο σώμα του ασθενή, όπου και λαμβάνουν σήμα από την κάψουλα. Με την συγκεκριμένη τεχνική η μέση τιμή σφάλματος της πραγματικής
θέσης της κάμερας ανέρχεται σε 37.7 mm, με μέγιστη τιμή τα 114 mm. \par
Ωστόσο, η συγκεκριμένη τεχνική αδυνατεί να εντοπίσει με ακρίβεια μικρές αλλοιώσεις του βλεννογόνου του εντέρου, 
εξαιτίας του μικρού χρόνου μετάβασης της κάψουλας στο εσωτερικού του πεπτικού σωλήνα ή της πολυποίκιλης ανατομίας, με αποτέλεσμα να εκπέμπει αλλοιωμένα δεδομένα. 
Με σκοπό τη βελτίωση εντοπισμού της κάψουλας, διάφορες τεχνικές, βασισμένες σε μαγνητικούς αισθητήρες, παρέχουν αποτελέσματα με σημαντικά μικρότερα σφάλματα, 
με μέση τιμή σφάλματος της θέσης στα 3.3 mm και μέση γωνιακή απόκλιση τις $3^{\circ}$. Παρ' όλα αυτά, κατά το πόσο είναι ακριβής η μαγνητική τεχνική εντοπισμού εξαρτάται σε
πολύ μεγάλο βαθμό από το πλήθος των εξωτερικών αισθητήρων, που χρησιμοποιούνται.\par
Ένα ταξίδι της κάμερας διάρκειας 8 ωρών μπορεί να παράξει περίπου 50.000 εικόνες-καρέ και ο χρόνος που απαιτείται από έμπειρους γαστρεντερολόγους για να τις μελετήσουν,
με τη βοήθεια εξειδικευμένου εμπορικού λογισμικού, κυμαίνεται από 45 λεπτά της ώρας μέχρι αρκετές ώρες.
Η ποιότητα των WCE εικόνων-καρέ είναι κατά κύριο λόγο χαμηλή, ενώ το ποσοστό εντοπισμού σημαντικών κλινικών ευρημάτων από τους ειδικούς
εκτιμάται ότι είναι περίπου 40\%.\par
Με σκοπό να βελτιωθεί η απόδοση και η αναγνώριση των εικόνων WCE, έχουν προταθεί αρκετές διαφορετικές μέθοδοι. Σε αυτές 
περιλαμβάνονται μέθοδοι συνοπτικής παρουσίασης των βίντεο, προκειμένου να αναδειχθούν σημεία ενδιαφέροντος από τις εικόνες, με τελικό αποτέλεσμα 
την επιλεκτική παρουσίαση εικόνων-καρέ, που παρουσιάζουν πραγματικό ενδιαφέρον, βάση του περιεχόμενού τους~\cite{zheng2012detection}. 




\section{Σκοπός της εργασίας}

Σκοπός της παρούσας εργασίας είναι η διερεύνηση και η αξιολόγηση διαφορετικών τεχνικών περιγραφής χαρακτηριστικών, οι οποίες εντοπίζουν και εξάγουν σημεία ενδιαφέροντος,
σε συνδυασμό με μια μέθοδο εύρεσης ομογραφίας, σε πεπερασμένη σειρά εικόνων (καρέ βίντεο),
που είναι τεχνητά τροποποιημένες, με βάση την κλίμακα και την γωνία περιστροφής τους. 
Οι τροποποιήσεις, που εφαρμόζονται στις εικόνες-καρέ, εξομοιώνουν τις τυπικές μεταβολές της κάμερας κατά την πορεία της διαδρομής της.
Οι παραπάνω τεχνικές παρουσιάζονται και αναλύονται εκτενώς στα επόμενα κεφάλαια. Στην περίπτωση των περιγραφέων σημείων ενδιαφέροντος, 
αναλύεται η επαναληπτικότητα των σημείων σε ζευγάρια εικόνων-καρέ (αυθεντικού και τροποποιημένου).\par
Συγκεκριμένα θα διερευνηθούν οι εξής περιγραφείς:

\begin{itemize}
 \item MSER (Maximally Stable External Regions)
 \item SIFT (Scale Invariant Feature Transform)
 \item LIOP (Local Invariant Order Patern)
 \item EIGEN (Good Features to Track) + FREAK (Fast Retina Keypoints)
 \item FAST (Fusing Points and Lines for High Performance Tracking) + FREAK (Fast Retina Keypoints)
 \item SURF (Speeded Up Robust Features)
\end{itemize}


% ``αλλο"
% \newpage
% 
% 
% \subsection{Δείγματα καρέ σημείων MSER}
% 
% Στα Σχήματα~\ref{fig:mser_healthy_figs} και~\ref{fig:mser_dis_figs} παρουσιάζονται τα σημεία ενδιαφέροντος βάσει του περιγραφέα MSER σε αυθεντικό και τροποποιημένο καρέ. Ενώ στο τελευταίο καρέ 
% γίνεται το ταίριασμα των δύο προηγούμενων, με την μετατόπιση των σημείων ενδιαφέροντος.\\
% 
% \textbf{Ομάδα 1 (φυσιολογικό)}
% 
% 
% \begin{figure}[!ht]
% \begin{minipage}[b]{0.5\linewidth}
% \centering
% \includegraphics[height=7.5cm]{../attachments/mser1_c.png}
% \caption*{(A)}
% 
% \end{minipage}
% \hspace{0.9cm}
% \begin{minipage}[b]{0.5\linewidth}
% \centering
% \includegraphics[height=7.5cm]{../attachments/mser1_d.png}
% \caption*{(B)}
% 
% \end{minipage}
% \end{figure} 
% 
% 
% \begin{figure}[!h]
% \begin{minipage}[b]{1.0\linewidth}
% \centering
% \includegraphics[scale=0.49]{../attachments/mser1_e.png}
% \caption*{(Γ)}
% 
% \end{minipage}
% \caption{Στα Σχήματα Γ, Δ, Ε απεικονίζονται με την εξής σειρά τα σημεία ενδιαφέροντος MSER στο αυθεντικό καρέ χωρίς έλκος, στο τροποποιημένο και το ταίριασμα των μετασχηματισμών στο τελευταίο.}
% \label{fig:mser_healthy_figs}
% \end{figure} 
%  
%  
% \newpage


% \begin{figure}[!ht]
% 
% \begin{minipage}[b]{0.8\linewidth}
% \centering
% \includegraphics[height=4cm]{../attachments/mser_cat1.png}
% \end{minipage}
% \hspace{1cm}
% \begin{minipage}[b]{0.9\linewidth}
% \centering
% \includegraphics[height=4cm]{../attachments/mser_cat2.png}
% \end{minipage}
% \end{figure} 

% Οι γωνίες είναι σημεία της εικόνας που παρουσιάζουν υψηλότερες μεταβολές έντασης της εικόνας, σε περισσότερες από μία κατευθύνσεις. Σχήμα~\ref{fig:eigen_fig1}. \\
% Η μεταβολή της έντασης κατά μήκος μιας κατεύθυνσης μπορεί να οριστεί από το σύνολο των τετραγωνισμένων διαφορών στην εικόνα (sum-of-squared-difference, SSD),πράγμα που απoσκοπεί στη συσχέτιση των συνεχόμενων καρε. \cite{changcorner}
% δηλαδή:
% \begin{equation}
% \label{eq1}
%   D(u,v) = \sum_{i,j} (I(i+u,j+v)-I(i,j))^{2} 
% \end{equation}
% 
% Σχήμα~\ref{fig:eigen_fig2}
%   
% % {\Large$ D(u,v) = \sum_{i,j}(I(i+u,j+v)-I(i,j))^{2} $  }  \textbf(1)  %Εικονα 2.5
%    % http://people.scs.carleton.ca/~c_shu/Courses/comp4900d/notes/lect9_corner.pdf 
%    
%    Επισημαίνουμε ότι
%  \begin{itemize}
%     
% 
%  \item Αν το σημείο της εικόνας είναι μια περιοχή προσωρινής εντάσεως, και οι δύο ιδιοτιμές θα είναι πολύ μικρές.
%  \item Αν περιέχει μία ακμή, τότε θα υπάρχει μία μεγάλη και μία μικρή ιδιοτιμή.
%  \item Αν περιέχει ακμές σε δύο ή περισσότερους προσανατολισμούς(λ.χ γωνία), τότε θα υπάρχουν δύο μεγάλες ιδιοτιμές.
%  
% \end{itemize}
%  
%  
% \newpage
% 
%  Αν οι τιμές u,v των διανυσμάτων είναι μικρές, από θεώρημα Taylor, στο (1) έχουμε: 
%  
% \begin{equation}
%     I(i+u,j+v) \approx I(i,j) + I_{x}u+I_{y}v  
% \end{equation}
%     
%     όπου
% \begin{equation}
%  I_{x} = \frac{\partial I}{\partial x} \text{ , } I_{y} = \frac{\partial I}{\partial y}  
% \end{equation}
% 
%  επομένως 
%  
% %%%%%%%%%%%%%%%%%%%%%% bgale ta sxolia %%%%%%%%%%%%%5
% 
% \begin{align}
% \begin{split}
% \label{eq4}
%  (I(i+u,j+v)-I(i,j))^2 &= I(i,j)+I_xu+I_yv-I(i,j))^2 \\
%  &= (I_xu+I_yv)^2  \\
%  &= I_x^2u^2 + 2I_xI_yuv+I_y^2v^2 \\
%  &=  \begin{bmatrix}  u & v \end{bmatrix}
%  \begin{bmatrix} I_x^2 & I_xI_y \\ I_xI_y & I_y^2 \end{bmatrix}
%  \begin{bmatrix} u\\ v \end{bmatrix}  
% \end{split}
% \end{align}
% 
% \begin{equation}
% \label{eq5}
%  \ref{eq1}\xrightarrow{\ref{eq4}}  D(u,v) = \begin{bmatrix} u & v \end{bmatrix}   \begin{bmatrix} \sum I_x^2 & \sum I_xI_y \\ \sum I_xI_y & \sum I_y^2 \end{bmatrix}\begin{bmatrix} u\\ v \end{bmatrix} 
% \end{equation}
% 
% 
% 
%  \vspace{1cm}
% 
% Όπου \ref{eq5} είναι και η συνάρτηση της έλλειψης:
% \vspace{0.1cm}
% 
% \begin{equation} 
% C =  \begin{bmatrix} \sum I_x^2 & \sum I_xI_y \\ \sum I_xI_y & \sum I_y^2 \end{bmatrix}\begin{bmatrix} u\\ v \end{bmatrix} =   \begin{bmatrix} \lambda_1 & 0 \\ 0 & \lambda_2 \end{bmatrix}
% \end{equation}
% 
%  
% \vspace{1.5cm}
%  
% Ο πίνακας C απεικονίζει την αλλαγή της εντάσεων σε μια συγκεκριμένη διεύθυνση. Δηλαδή τις επικρατέστερες
% τιμές των διευθύνσεων που ευθυγραμμίζονται με τον άξονα x ή ψ.\\ 
% Αν κάποιο από τα $\lambda$ είναι κοντά στο 0, τότε δεν αποτελεί γωνία.



