\section{Συνεισφορά}

Στο πλαίσιο της διπλωματικής εργασίας παρουσιάζεται η αξιολόγηση διαφορετικών περιγραφέων σημαντικών σημείων, σε μια σειρά από
WCE βίντεο (διαδοχικών εικόνων-καρέ), κατά την ανάκτηση τεχνητών γεωμετρικών μετασχηματισμών (γωνίας περιστροφής και κλίμακας), που εφαρμόστηκαν 
στα αυθεντικά καρέ. Αρχικά, μελετήθηκε η βιβλιογραφία γύρω από τα WCE βίντεο, ο πρακτικός σκοπός τους, καθώς και η βιβλιογραφία 
σχετικά με τους περιγραφείς εικόνων, οι τεχνικές με τις οποίες λειτουργούν πάνω σε μία εικόνα, ώστε να 
εντοπιστούν σημαντικά σημεία ενδιαφέροντος. Συμπεριλήφθηκαν βιβλία, επιστημονικές δημοσιεύσεις, ιστοσελίδες και ακαδημαϊκές παρουσιάσεις διεθνών ινστιτούτων
σχετικές με τον τρόπο λειτουργίας και το θεωρητικό υπόβαθρο των περιγραφέων και των αλγορίθμων.
 Με βάση το θεωρητικό υπόβαθρο, μελετήθηκαν και υλοποιήθηκαν οι διαφορετικές τεχνικές για τον κάθε περιγραφέα. Δημιουργήθηκε μια εφαρμογή που υλοποιεί
 τους περιγραφείς, καθώς και παράλληλες εφαρμογές μέτρησης σφάλματος και δημιουργίας γραφημάτων, που οδήγησαν σε μία εκτεταμένη διαδικασία πειραμάτων,
 των WCE βίντεο της έρευνας. Τα εξαγόμενα αποτελέσματα, που περιλαμβάνουν τις ανακτηθέντες γεωμετρικές τιμές, συγκρίθηκαν μεταξύ τους, με 
 όλους τους πιθανούς συνδυασμούς γωνίας και κλίμακας, για την μέτρηση του μέσου σφάλματος, την αναπαράστασή του σε αυτόνομα γραφήματα,
 την επιτυχία διεκπεραίωσης των περιγραφέων τόσο σε ποσοστό εξαγωγής αποτελέσματος όσο και σε χρόνο διεκπεραίωσης.
 Η κύρια εφαρμογή μπορεί να χρησιμοποιηθεί στο μέλλον ως βάση (testbed) σε νέες εικόνες, με σκοπό την ανάκτηση των προαναφερθέντα τεχνητών γεωμετρικών μετασχηματισμών.
 
 \section{Συμπεράσματα}


 %% surf
Ο περιγραφέας SURF, όπως αναφέρθηκε στο προηγούμενο κεφάλαιο, αποτελεί ξεχωριστή περίπτωση, λόγω της πληθώρας των συνδυασμών των εσωτερικών μεταβλητών του. 
Παρατηρείται ότι, για όλες τις πειραματικές γωνίες, το μέσο σφάλμα ανάκτησης της κλίμακας μειώνεται. Κατά συνέπεια, αυξάνεταιη αποδοτικότητα των σημείων SURF, όσο αυξάνεται η τιμή 
των ζευγών οκτάβας-κλίμακας.\par
Τα γραφημάτων αναδεικνύουν με μεγαλύτερη ακρίβεια τη συμπεριφορά των διαφόρων συνδυασμών οκτάβας-κλίμακας, οι οποίοι σχεδόν εξολοκλήρου, σε μικρές μεταβάσεις κλίμακας, ακουλουθούν κοινή απόδοση. 
Ωστόσο σε μεγαλύτερες μεταβολές κλίμακας, για μεγέθη άνω του ποσοστού 120\%, τα ζεύγη που κρατούν την απόδοση σε σταθερά επίπεδα είναι εκείνα με την μέγιστη τιμή οκτάβας 
και συνήθως με την μέγιστη τιμή κλίμακας, ανεξάρτητα από τη τιμή της κλίμακας αναφοράς.
Εν συνεχεία, για τις πειραματικές κλίμακες, το μέσο σφάλμα ανάκτησης γωνίας είναι σχεδόν αμελητέο για την πλειονότητα των συνδυασμών οκτάβας-κλίμακας προς την αντίστοιχη κλίμακα αναφοράς. Πιο συγκεκριμένα, 
με εξαίρεση τις πολύ μικρές κλίμακες αναφοράς, όλοι οι συνδυασμοί οκτάβας-κλίμακας δεν δίνουν τα βέλτιστα αποτελέσματα. Αντίθετα, ο περιγραφέας λειτουργεί με πολύ καλά αποτελέσματα 
στις μικρομεσαίες κλίμακες μέχρι τις μέγιστες της αναφοράς. Παρατηρείται όμως, ότι σε μεγαλύτερες κλίμακες, ο αριθμός των μικρών σφάλματων περιορίζεται, καθώς αυξάνεται η τιμή των ζευγών οκτάβας-κλίμακας.
Τέλος, ο SURF παρουσιάζει εξαιρετικά ποσοστά επιτυχίας ανάκτησης, ενώ παράλληλα επιτυγχάνει υψηλές ταχύτητες διεκπεραίωσης και ταιριάσματος.\par

%% sift
Σε παρόμοιο ή ακόμα καλύτερο πλαίσιο απόδοσης κινείται ο SIFT, που κινείται σε εξαιρετικά επίπεδα απόδοσης σε όλο το εύρος των πειραμάτων. Όμως ο SIFT δείχνει να υστερεί αρκετά χρονικά ως προς την ταχύτητα διεκπεραίωσης, σε ακραίες τιμές της κλίμακας.\par

%% liop
Αρκετά κοντά στις επιδόσεις του SURF και του SIFT κινούνται τα σημεία LIOP, που καταφέρνουν να επιτύχουν το ίδιο χαμηλές τιμές σφάλματος κλίμακας, σε όλο το εύρος των γωνιών. Σε σύγκριση με τους προαναφερθέντες τα σημεία LIOP, έχουν
καλά ποσοστά επιτυχίας εξαγωγής περιγραφέων τόσο βάσει γωνίας, όσο και κλίμακας. Τα σημεία LIOP παρουσιάζουν χαμηλά ποσοστά και υψηλότερες τιμές σφάλματος μόνο στις πολύ μικρές τιμές της κλίμακας. Ακόμα και σε υψηλές τιμές κλίμακας, 
οι τιμές του σφάλματος κινούνται σε παρόμοια χαμηλά επίπεδα. Τέλος, μειονέκτημα των σημείων LIOP είναι ο πολύ μεγάλος χρόνος διεκπεραίωσης σε όλο το εύρος των πειραμάτων, ενώ αντίθετα ο RANSAC καταναλώνει το λιγότερο χρόνο στην αντιστοίχιση
των σημείων ενδιαφέροντος. \par

%%mser
Στη συνέχεια, ο περιγραφέας MSER κινείται επίσης σε χαμηλές τιμές σφάλματος, κλίμακας και γωνίας. Σε σχέση με όλους τους περιγραφείς και σε αντίθεση με τους προαναφερθέντες, επιτυγχάνει το βέλτιστο χρόνο διεκπεραίωσης, 
σε ακραίες τιμές της κλίμακας, ενώ σε μικρές κλίμακες είναι ταχύτατος. Εν αντιθέσει, έχει μέτρια ποσοστά επιτυχίας εξαγωγής περιγραφέων.

%% fast & eigen
Ο FAST εμφανίζει υψηλά αριθμητικά σφάλματα, ενώ ο αλγόριθμος EIGEN παρουσιάζει ακόμα υψηλότερα. Σχετικά με την ανάκτηση κλίμακας, οι τιμές του EIGEN, θα μπορούσαν να είναι ανεκτές, σε περιπτώσεις που η κλίμακα είναι χαμηλή ως μέτρια. Σε υψηλότερες τιμές 
γωνίας, ο EIGEN αδυνατεί να εξάγει αποτελέσματα. Σε ανάκτηση κλίμακας, οι τιμές σφάλματος ξεφεύγουν από τα αποδεκτά επίπεδα και θεωρούνται πως αδυνατούν πλήρως να αποδώσουν ικανοποιητικά με παράλληλα φτωχά ποσοστά επιτυχίας.
Οι EIGEN και ο FAST δείχνουν να μπορούν να αποδώσουν μόνο στις αυθεντικές διαστάσεις των καρέ-εικόνων, ενώ έχουν υψηλούς χρόνους διεκπεραίωσης.\par


Καταλήγοντας στο βέλτιστο περιγραφέα, τίθενται ζητήματα πολιτικής της εκάστοτε εφαρμογής. Αν κάποια εφαρμογή είναι ανεκτική σε σφάλματα και ο χρόνος διεκπεραίωσης είναι κρίσιμο ζήτημα, 
τότε ο MSER είναι ο κατάλληλος περιγραφέας.
Στην περίπτωση, που η ακρίβεια σφάλματος και η υψηλή σταθερότητα σε ανάκτηση είναι το μείζον ζήτημα και υπάρχει ανεκτικότητα σε χρόνο, ο περιγραφέας SURF αποτελεί την πιο ισορροπημένη και βέλτιστη επιλογή, 
πράγμα που επιβεβαιώνει και τη φήμη του σε παρόμοιες εφαρμογές. \par

Αξίζει να σημειωθεί ότι η πρωτοτυπία στο πλαίσιο της πτυχιακής, είναι ότι καταφέραμε να διερευνήσουμε με εξαντλητικό τρόπο όλους τους συνδυασμούς των παραμέτρων του SURF. Εξ όσων γνωρίζουμε δεν υπάρχει στη βιβλιογραφία 
αντίστοιχη μελέτη όσον αφορά τη βέλτιστη χρήση των SURF, σε δεδομένα που έχουν προέλθει από ενδοσκοπικές κάψουλες.



\section{Μελλοντικές Επεκτάσεις}

Η επέκταση της εφαρμογής, με περισσότερους περιγραφείς και αλγορίθμους, όπως ο BRISK~\cite{leutenegger2011brisk}, ο ORB~\cite{rublee2011orb} και ο BRIEF~\cite{calonder2010brief}, μπορεί να αποτελεί ένα άμεσα μελλοντικό βήμα, 
ώστε να διευρυνθεί το εύρος των διαθέσιμων αξιολογήσεων.\\
Επίσης, θα μπορούσε να διερευνηθεί το περιοχόμενο της εργασίας που περιγράφτηκε και σε άλλους βιοϊατρικούς τομείς, όπου υπάρχει ανάγκη παραγωγής αντίστοιχων βίντεο.\\
Τέλος, όλοι οι προαναφερθέντες περιγραφείς και αλγόριθμοι, εφόσον εκτελούνται σε κάποιο υπολογιστικό σύστημα, είναι πολύ σημαντικό 
να υλοποιούνται με ταχύτερες γλώσσες προγραμματισμού, που να πληρούν τη φορητότητα, αλλά και τη βέλτιστη χρήση των υπολογιστικών πόρων.

















