% print no page number
\thispagestyle{empty}

% The abstract in greek

\begin{center}
\Large
{\bf ΕΥΧΑΡΙΣΤΙΕΣ}\\[15mm]
\end{center}

Η παρούσα διπλωματική εργασία εκπονήθηκε κατά το ακαδημαϊκό έτος 2013-2014. Θα ήθελα να ευχαριστήσω θερμά τον επιβλέποντα επιστημονικό συνεργάτη κ. Ευάγγελο Σπύρου για την
ευκαιρία που μου έδωσε αναθέτοντάς μου μια εργασία ερευνητικού περιεχομένου, καθώς και την αδιάκοπη και άμεση καθοδήγηση στην συγγραφή, 
τις συμβουλές και τις υποδείξεις του, καθ' όλη τη διάρκεια της.
Τέλος, κυρίως την οικογένεια μου, καθώς και τους στενούς μου φιλικούς κύκλους για τη στήριξή τους, σε πνευματικό και υλικό επίπεδο, καταλυτικό για την ολοκλήρωση της εργασίας. \\ \\Σταύρος Νιάφας\\
Σεπτέμβρης 2014

\newpage


\begin{center}
\Large
{\bf ΠΕΡΙΛΗΨΗ}\\[15mm]
\end{center}

Στην παρούσα πτυχιακή εργασία διερευνώνται και υλοποιούνται μέθοδοι εξαγωγής και παρακολούθησης χαρακτηριστικών σε wce βίντεο. Στo πλαίσιo της εργασίας, 
τα δεδομένα προέρχονται από μια ειδική ασύρματη έγχρωμη κάμερα, σε μέγεθος μεγάλης βιταμινούχου κάψουλας, που χορηγείται προς κατάποση από τον εκάστοτε εξεταζόμενο.
Η κάμερα αυτή προωθείται κατά μήκος του πεπτικού σωλήνα του εξεταζόμενου με τις περισταλτικές κινήσεις του εντέρου. Κατά το ταξίδι της στο γαστρεντερικό σωλήνα παράγονται βίντεο, 
που αποτελούνται από όλα τα καρέ που συλλαμβάνει η κάμερα (WCE).
Από αυτά τα βίντεο εξάγονται εικόνες, στις οποίες και εφαρμόζονται διαφορετικές τεχνικές εξαγωγής και περιγραφής χαρακτηριστικών σημείων ενδιαφέροντος. Για την αξιολόγηση των διαφορετικών τεχνικών εξαγωγής χαρακτηριστικών,
κατασκευάζονται τεχνητά καρέ, που προκύπτουν από την εφαρμογή απλών μετασχηματισμών στα εξαχθέντα καρέ.

% leave 50mm empty space below
\vspace{50mm}

% The abstract in english

%~\cite{spyrou2005fusing}